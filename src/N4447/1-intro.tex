\rSec0[revision]{Revision history}
\begin{itemize}
\item Change the layout of the document to more C++ standard
\item Changed the meaning of \texttt{typename<T, C>} to type-list
\item Introduce \texttt{typeid<T, C>} in replacement of \texttt{typename<T, C>}
\item Changed \tcode{requires} by \tcode{,}
\item Address private access problem
\item Explain more why is important to reflect namespaces and modules
\item Resume some parts ( but still a big document )
\item Changes in impact
\item Merged with attribute proposal N3984 ( typed attributes )
\item Merged with new type traits proposal N3987 ( type traits )
\end{itemize}
\rSec0[intro]{Intro}
\begin{enumerate}
\item Overview of the new three instructions:
\begin{itemize}
\item \texttt{typedef<T, C>} that expands the definition of type T into a parameter back instantiation. 
\item \texttt{typename<T, C>} that expands types of the definition of type T into a parameter back. 
\item \texttt{typeid<T, C>} that expands members names of type T into a parameter back instantiation. 
\end{itemize}
\item Simple Sample Code
\begin{codeblock}
using namespace std;
namespace ns {
 struct X {
  int x, y;
 };
}

std::vector<std::string> names{ typeid<ns::X, is_member_object_pointer>... };
std::tuple<typename<ns::X, is_member_object_pointer>...>
     mytuple = std::make_tuple(typedef<ns::X, is_member_object_pointer>...);

\end{codeblock}
\item Compiler will turn the last two lines into this:
\begin{codeblock}
vector<string> names { "x","y" };
std::tuple<ns::X::int, ns::X::int>
       mytuple = std::make_tuple(
                &ns::some_struct::x,
                &ns::some_struct::y);
\end{codeblock}
\item What it's not:
\begin{itemize}
	\item we don't need no new keywords
	\item we don't need no AST control
	\item no magic namespaces or runtime structs
	\item not an as-if-lib approach
	\item all we need by now just another syntax keywords
\end{itemize}
\item That's it.
\end{enumerate}